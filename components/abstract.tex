% Abstract for the TUM report document
% Included by MAIN.TEX


\clearemptydoublepage
\phantomsection
\addcontentsline{toc}{chapter}{Abstract}	





\vspace*{2cm}
\begin{center}
{\Large \bf Abstract}
\end{center}
\vspace{1cm}

This bachelor thesis describes the implementation of an Android Framework to access mass storage devices over the USB interface of a smartphone. First the basics about USB (i.e. interfaces, endpoints and USB On the go) and accessing USB devices via the official Android API is discussed. Next the USB mass storage class is explained. It was designed by the USB-IF to access mobile mass storage like USB pen drives or external HDDs. For communication with mass storage devices, most important are the bulk-only transfer and the SCSI transparent command set. Furthermore file systems, for accessing directories and files, are illustrated. This thesis focuses on the FAT32 file system from Microsoft, because it is the most commonly used file system on such devices.

After the theory it is time to look at the implementation of the Framework. In this section, the first concern is the purpose in general. Then the architecture of the framework and the actual implementation are presented. Important parts are discussed in detail.

The thesis finishes with an overview of the test results on various Android devices, a short conclusion and an outlook to future developments. Moreover the current status of the developed framework is visualized. 