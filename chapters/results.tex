\chapter{Summary}

The goal of the thesis was to develop a framework, for the Android operating system, which allows access to USB mass storage devices. These devices include USB pen drives, card readers or even external HDDs. The framework allows discovering and exchanging data with these devices in terms of directories and files.

The thesis gives an overview of basic aspects relating USB in general and how to use the Android USB Host API. A very important aspect is the description of the USB mass storage class, including the bulk-only transfer and the SCSI transparent command set. The most important SCSI commands were also introduced. After that a detailed view on the theory about file systems, in detail the FAT32 file system, follows.

After the theoretical part, the developed framework is described, it's purpose, general structure and most important aspects in detail.

The part about the quality management gives a short overview of the interplay of the framework with different devices and Android versions.

\section{Current status}

The Framework works on every device with Android 3.1 or later, which has hardware and software enabled USB host support. Every Android device which meets these requirements should be supported. The framework currently supports mass storage devices using the bulk-only transport with the SCSI transparent command set. The mass storage device must be formatted with an MBR, located at the beginning of the storage. There is no support for other partition tables. Devices completely without an partition table are also unsupported. The supported file system is FAT32, which is the most commonly used for USB mass storage devices. Other file systems, even the one from the FAT family, like FAT12 or FAT16, are not supported.

Despite these limitations the framework has all features which had been determined at the beginning of this thesis\footnote{The desired requirements and features are described in section \ref{implementation_purpose}.}. In the source code, there are currently some TODOs to mark starting points for minor enhancements. But these points do not compromise the everyday use. Currently there are a lot of debug log messages which help to understand the operation of the framework and may decrease the performance, especially when reading and writing huge amounts of data from or to files.

For example, the SCSI REQUEST SENSE command could be added. Further information about an unsuccessful command can be acquired by it. This is not a serious problem, because it is very rare that a device cannot execute a command successfully. During the development, this constellation occurred only when the commands transfered to the device were incorrect and thus the framework had a bug. For the development of extensions to the framework, the addition of this command may be helpful.

Nevertheless, the framework is very easy to extend. Most parts operate independent from each other and can be easily be exchanged. Other block device drivers, partition tables or file systems can easily be added without changing unrelated parts. This is why the framework often relies on interfaces instead of particular implementations and often uses factory classes for the initialization of instances.

\section{Conclusion}

Developing Android Apps, is surely much fun. The Android API is well structured, organized and easy to understand. This also stays on when developing advanced applications like this one. 

Very interesting was the aspect of developing low level things like a block device driver and a file system driver in the Java programming language. Normally these things are done in C, and not in a higher level language like Java. But solely relying on Java and strictly avoiding C code was not a problem at any time, Java did the job very well! This shows that Java is also perfectly capable of bringing the object oriented approach to things located at lower levels in an operating system or the kernel.

The documentation on the USB mass storage class and the SCSI transparent command set is very rare and often complex. It needs a lot of foreknowledge on some topics. Nevertheless after consulting various resources and spending a lot of time reading, the comprehension continuously increases.

\chapter{Outlook}

The developed framework is usable perfectly at its current status. As always there is room for further development and features. Integrating other block device drivers and file systems are features that come instantly in mind. Maybe someday someone likes to connect his external CD/DVD drive to his Android device?

But not only other block device and file system drivers are possible. Currently the framework has no intelligent caching mechanisms. However it always writes the changes directly to the storage which may be inefficient sometimes. Reading data, at the moment, is implemented straight forward. The appropriate data is just read from disk, there are no special strategies like reading ahead or guessing what data the user wants to access next. These are all things which are available in every up to date operating system and many people invested a lot of time in efficient strategies for caching, etc. Maybe implementing such techniques increase the user experience.

Another useful extension to the framework could be the integration into the new FileSystem API of Java 7\footnote{\url{http://docs.oracle.com/javase/7/docs/technotes/guides/io/fsp/filesystemprovider.html}}. This would give the user the ability to use the default Java API for accessing files. Another benefit could be the use of pipes to integrate the mass storage device in the internal file system of Android. The framework would then run in background and listen for events relating the piped directories and files. Every change to the piped structure would then be written back to the actual mass storage device. With this solution the mass storage device is mirrored into the Android file system.

Maybe with the next Android version Google adds native support for USB mass storage devices in their stock Android version while making this framework (nearly) obsolete. Some manufacturers already noticed that this feature is pretty useful especially when looking at the latest trend omitting a slot for micro sd cards and relying only on the internal storage of a device. But we will see what Google brings next!