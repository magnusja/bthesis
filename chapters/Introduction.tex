\chapter{Introduction}
\label{chapter:Introduction}



Since Android 3.1 a lot of Android devices come with USB host support (USB on the go). That means a normal Android tablet or phone can not only act as a USB client when connected to a computer. It can also act as a USB host for periphals by powering the bus with the needed 5 Volt and changing into a USB host mode and enumerating connected USB devices\cite{android_usb_host}. Android currently supports interrupt, bulk and control transfers (isochronous transfers are currently unsupported). That means nearly every USB device can, theoretically, be used with an Android device (Webcams or Audio devices mostly use isochronous transfers and can thus not be used at the moment). The Android host API easily allows to communicate with connected USB devices, meaning you can write your own high level USB driver in Java.\\
Thus the idea of connecting a USB mass storage device like USB sticks or external HDDs is not that far away. Especially when you look at recent occurences where a lot of devices miss a slot for external SD-Cards and only offer a solid internal storage. Unfortunately the stock Android comes without support for USB storage devices. That means when you are connecting your mass storage device nothing will happen. You can not access the data via a file manager or something similiar. On rooted devices this is possible because the alternative Android ROMs provide support for it. But with the Android USB Host API it should also be possible to access such devices with out rooting the device and flashing an alternative ROM. The only thing you have to do is to implement the low level USB communication via eg. bulk transfers and the abtraction of directorys and files via a filesystem. This can of course also be done in Java and not only in plain C.\\
Currently there are two applications in the Google Play Store which allow accessing mass storage devices without root rights! First there is a plugin for the Total Commander called USB-Stick Plugin-TC. The plugin extends the Total Commander application by USB mass storage access. It currently supports FAT12, FAT16, FAT32 exFAT and NTFS (read only). There is a free trial version available. The second application is called Nexus Media Importer. It supports FAT16, FAT32 and NTFS (also read only). There is no free trial verson available. In general both apps support USB sticks, external HDDs and card readers.\\
The problem both applications have is that there is no solution to access the mass storage from other apps. That means all accessed data has to be cached and copied to the internal storage before any other app can access it. These limitations are annoying but it seems that they are impossible to overcome.\\
Both applications are proprietary and thus do not offer the ability to look into and change the source code. Therefore an open source Android Framework for accessing mass storage devices is developed in this bachelor thesis. The license is the very liberale Apache License, Version 2.0, Android is also licensed under.\\
Due to the same license it would be possible that Google integrates this solution into the official Android. Indeed there are some disadvantages which makes the integration unlikely. First all needed things, like filesystems (eg. FAT32) or the SCSI transparent command set, for mounting USB mass storage are already implemented in the underlying Linux kernel. Google just deactivated the support for it. Next with our solution only apps which use our framework can access USB storage devices. It would be much nicer if the conncected devices would be mounted in the normal unix filesystem like sd cards. For example under /mnt/usbstick0. This would make it possible for other apps easily access data from USB mass storage without extra changes to the application. Due to this I think that is very unlikely that Google will integrate this framework into the official Android!
 
\section{Basics about USB}

USB means universal serial bus interface is a serial bussystem to connect a computer with external devices. The first version was introduced by Intel and now the specification is done by the USB foundation. In USB there exists one USB host controller (eg. computer) which initiates the connection and comunication and multiple clients (slaves). 